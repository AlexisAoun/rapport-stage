\documentclass[12pt]{article}
\usepackage{helvet}
\renewcommand{\familydefault}{\sfdefault}
\usepackage[letterpaper, top=1.2in, bottom=1.2in, left=1.2in, right=1.2in, heightrounded]{geometry}
\linespread {1.15}
\usepackage{graphicx}
\graphicspath{ {./assets/img} }
\author {Alexis Aoun}
\begin{document}
\begin {sloppypar}
\title {Stage Atos}
\date {}
\maketitle
\newpage

% 2 - missions
% MPP
\section {Objectifs et réalisation de mes missions}
\subsection {MPP Dashboard}
\subsubsection {Le contexte}
\paragraph {}
Atos a une plateforme en ligne, My Atos, sur laquelle tous les employés doivent 
renseigner leurs informations personnelles ainsi que leurs parcours professionel et 
académique. Le problème de cette procédure, qui est en théorie obligatoire, est qu' 
une partie importante des salariés ne remplissent pas ces informations, et la plupart 
du temps cela est dû à un simple oubli. Pour y remédier, le service RH et la direction 
général d'Atos décidèrent de lancer un projet interne qui permetterait de relancer 
les employés automatiquement à partir d'un fichier excel contenant les informations 
de tous les collaborateurs, fournit par le service RH. Ce projet fut baptisé MPP Dashboard.

\paragraph {} 
Le projet est une application web(TODO) dont le fonctionnement repose sur trois processus
principaux : 
\begin {enumerate}
  \item 
    Le traitement de l'excel : Le service RH fournit un fichier excel sous format XSLX 
    (Le format par défaut des fichiers excels microsoft). Celui-ci doit être traité de 
    manière à ce que les informations qu'il contient puissent être manipuler 
    programmatiquement.
  \item 
    Le filtrage : Une fois les donnés des collaborateurs dans notre application, on
    sauvegarde dans une base de donnée uniquement ceux dont le taux de renseignement 
    d'informations et en-dessous d'un seuil défini par les administrateurs de 
    l'application. Le taux par défaut est de 90\%. 
  \item 
    L'envoie de mail : L'application enverra des mails à tous les salariés présent 
    dans la base de données après la phase de filtrage. Un chiffre correspondant au nombre
    de mails envoyés au collaborateurs est associé à chacun d'entre eux. Au bout du 
    5ème mail l'employé n'ayant pas renseigner les informations nécéssaires est signalé 
    automatiquement au service RH.
\end{enumerate}
\newpage
\begin{figure}
  \includegraphics[width=\textwidth] {mpp-diagram.png}
  \caption {Diagram du fonctionnement de MPP Dashboard}
\end{figure}
\paragraph {}
Les technologies utilisés pour la réalisation du projet sont : 
\begin {itemize}
\item   
  Pour la base de données : PostgreSQL (TODO)
\item 
  Pour le serveur backend (TODO) : ExpressJS (TODO)
\item 
  Pour l'interface grahique : ReactJS (TODO)
\end {itemize}

\subsubsection{La problèmatique}
Lorsque je suis arrivé sur le projet, celui-ci a déjà été développé mais avait un 
problème avec l'un des processus, celui d'extraction de l'excel. En effet la librairie (TODO)
utilisée pour remplir cette tâche, du nom d'Excellente, a été développée en interne par des 
employés d'Atos et comporte certains inconvenients : 
\begin{itemize}
  \item 
    Il comporte plusieurs bugs (TODO). Cela est notamment dû au fait que cette librairie 
    n'est pas connue du grand publique et ne reçoit donc pas un grand nombre de tests et 
    de contributions 
  \item 
    La librairie devient exponentiellement lente avec la grandeur du fichier excel. Hors
    celui qu'on a à extraire fait plus de 40 000 lignes
\end{itemize}
Ma tâche est donc de trouver une solution qui puisse répondre à la fois aux problèmes
de vitesse et de fiabilité.
\newpage
\subsubsection{Les solutions possibles}

\end{sloppypar}
\end{document}

